\documentclass[12pt,a4paper]{article}

% Pacotes para o português.
\usepackage[brazilian,provide=*]{babel}
\usepackage[utf8]{inputenc}
\usepackage[T1]{fontenc}
\usepackage{graphicx}
\usepackage{xcolor}
\usepackage{longtable}
\usepackage{indentfirst}
\usepackage{url}
\usepackage{array}
\usepackage[top=2.5cm, bottom=2.5cm, left=2.5cm, right=2.5cm]{geometry}
\usepackage{multirow}
\usepackage{amssymb}
\usepackage{amsmath}
\usepackage{caption}
\usepackage{setspace}
\usepackage{breakcites}
\usepackage{float}
\usepackage{times}
\usepackage{lipsum}
\usepackage{booktabs}

\usepackage{tikz}
\usetikzlibrary{shapes.multipart, arrows.meta, positioning, fit}

\usepackage{sectsty}
\usepackage[compact]{titlesec}

\sectionfont{\normalfont\normalsize\bfseries}
\subsectionfont{\normalfont\normalsize\bfseries}
\subsubsectionfont{\normalfont\normalsize\bfseries}

% Comando para marcar o texto para revisão.
\newcommand{\rev}[1]{\textcolor{red}{#1}}

% Permite escrever aspas normais "text" em vez de ``text''
\usepackage[autostyle]{csquotes}
\MakeOuterQuote{"}

\begin{document}

\begin{titlepage}
	\begin{center}
	
	\vspace{115pt}
    \textbf{\Huge{Manual do Usuário - Simple C}}\\
        
	\vspace{115pt}
    Felipe Gomes da Silva \\
    Luis Henrique Salomão Lobato \\
	\end{center}
	
	\vspace{1cm}
	\begin{center}
		\vspace{\fill}
    \large{Setembro, 2025} 
	\end{center}
\end{titlepage}

% Table of contents

\tableofcontents
\newpage
\section{Introdução}
Este manual do usuário fornece uma visão geral do Simple C, uma linguagem de programação que visa simplicar o uso de conceitos fundamentais da linguagem C. O Simple C é projetado para ser fácil de aprender e usar, tornando-o ideal para iniciantes em programação, mas removendo algumas funcionalidades avançadas da linguagem C. 

Neste momento, abordaremos apenas o analisador léxico da linguagem. Utilizamos a ferramenta Flex para construir o analisador léxico, que é responsável por identificar e classificar os tokens na entrada do código-fonte.

\section{Testando o Analisador Léxico}

Para testar o analisador léxico do Simple C, siga os passos abaixo:
\begin{enumerate}
  \item Certifique-se de ter o Flex instalado em seu sistema. Você pode verificar isso executando o comando \texttt{flex --version} no terminal.
  \item Clone o repositório do Simple C do GitHub:
    \begin{verbatim}
      git clone github.com/felipe-gsilva/simple-c
    \end{verbatim}
  \item Navegue até o diretório do projeto:
    \begin{verbatim}
      cd simple-c/flex/
    \end{verbatim}
  \item Gere o analisador léxico usando o Flex e o compile com nosso Makefile:
    \begin{verbatim}
      make
    \end{verbatim}
  \item Execute o analisador léxico com um arquivo de entrada que contenha código Simple C:
    \begin{verbatim}
      ./simplec < input_file.c
    \end{verbatim}
    Substitua \texttt{input\_file.c} pelo caminho do arquivo que você deseja analisar.

\end{enumerate}

Foram criados alguns arquivos de teste na pasta \texttt{tests} do repositório. Você pode usar esses arquivos para verificar o funcionamento do analisador léxico. Cada arquivo de teste contém exemplos de código Simple C que abrangem diferentes aspectos da linguagem, como declarações de variáveis, estruturas de controle, funções, entre outros. 

Tenha em vista que o analisador sintático ainda não foi implementado, portanto, mudanças na gramática podem ocorrer futuramente.

\newpage
\section{Tokens Reconhecidos}
O analisador léxico do Simple C reconhece os seguintes tokens:

\begin{longtable}{lll}
\caption{Tabela de Tokens do Simple C}
\label{tab:tokens} \\

\toprule
\textbf{Nome do Token} & \textbf{Lexema(s)} & \textbf{Descrição} \\
\midrule
\endfirsthead

\toprule
\multicolumn{3}{l}{\small\textit{Tabela \ref{tab:tokens}: Tabela de Tokens do Simple C (Continuação)}} \\
\textbf{Nome do Token} & \textbf{Lexema(s)} & \textbf{Descrição} \\
\midrule
\endhead

\bottomrule
\endlastfoot

\multicolumn{3}{l}{\textbf{Palavras-chave: Tipos de Dados}} \\
\midrule
KEYWORD\_INT & \texttt{int} & Palavra-chave para tipo inteiro. \\
KEYWORD\_FLOAT & \texttt{float} & Palavra-chave para tipo ponto flutuante. \\
KEYWORD\_CHAR & \texttt{char} & Palavra-chave para tipo caractere. \\
KEYWORD\_STRING & \texttt{string} & Palavra-chave para tipo string (cadeia de caracteres). \\
KEYWORD\_VOID & \texttt{void} & Palavra-chave para tipo vazio/nulo. \\
KEYWORD\_BOOL & \texttt{bool} & Palavra-chave para tipo booleano. \\
\midrule
\multicolumn{3}{l}{\textbf{Palavras-chave: Controle de Fluxo e Comandos}} \\
\midrule
KEYWORD\_IF & \texttt{if} & Inicia uma estrutura condicional. \\
KEYWORD\_ELSE & \texttt{else} & Bloco alternativo de uma estrutura condicional. \\
KEYWORD\_FOR & \texttt{for} & Inicia um laço de repetição `for`. \\
KEYWORD\_WHILE & \texttt{while} & Inicia um laço de repetição `while`. \\
KEYWORD\_DO & \texttt{do} & Inicia um laço de repetição `do-while`. \\
KEYWORD\_SWITCH & \texttt{switch} & Inicia uma estrutura de seleção múltipla. \\
KEYWORD\_CASE & \texttt{case} & Define um rótulo dentro de um `switch`. \\
KEYWORD\_DEFAULT & \texttt{default} & Define o rótulo padrão de um `switch`. \\
KEYWORD\_BREAK & \texttt{break} & Interrompe a execução de um laço ou `switch`. \\
KEYWORD\_CONTINUE & \texttt{continue} & Pula para a próxima iteração de um laço. \\
KEYWORD\_RETURN & \texttt{return} & Retorna um valor de uma função. \\
\midrule
\multicolumn{3}{l}{\textbf{Identificadores e Literais}} \\
\midrule
IDENTIFICADOR & \texttt{var}, \texttt{\_x}, ... & Nome de variável, função, etc. \\
INT & \texttt{123}, \texttt{42} & Valor literal inteiro. \\
FLOAT & \texttt{3.14}, \texttt{0.5} & Valor literal de ponto flutuante. \\
CHAR & \texttt{'a'}, \texttt{'\textbackslash n'} & Valor literal de caractere. \\
STRING & \texttt{"hello"} & Valor literal de string. \\
BOOLEAN\_LITERAL & \texttt{true}, \texttt{false} & Valor literal booleano. \\
\midrule
\multicolumn{3}{l}{\textbf{Operadores}} \\
\midrule
OP\_SOMA & \texttt{+} & Operador de adição. \\
OP\_SUB & \texttt{-} & Operador de subtração. \\
OP\_MULT & \texttt{*} & Operador de multiplicação. \\
OP\_DIV & \texttt{/} & Operador de divisão. \\
OP\_MOD & \texttt{\%} & Operador de módulo (resto da divisão). \\
OP\_ATRIBUICAO & \texttt{=} & Operador de atribuição simples. \\
OP\_INC\_ATRIBUICAO & \texttt{+=} & Operador de atribuição com adição. \\
OP\_DEC\_ATRIBUICAO & \texttt{-=} & Operador de atribuição com subtração. \\
OP\_MULT\_ATRIBUICAO & \texttt{*=} & Operador de atribuição com multiplicação. \\
OP\_DIV\_ATRIBUICAO & \texttt{/=} & Operador de atribuição com divisão. \\
OP\_INC & \texttt{++} & Operador de incremento. \\
OP\_DEC & \texttt{--} & Operador de decremento. \\
OP\_IGUAL & \texttt{==} & Operador relacional de igualdade. \\
OP\_DIFERENTE & \texttt{!=} & Operador relacional de desigualdade. \\
OP\_MENOR & \texttt{<} & Operador relacional menor que. \\
OP\_MAIOR & \texttt{>} & Operador relacional maior que. \\
OP\_MENOR\_IGUAL & \texttt{<=} & Operador relacional menor ou igual que. \\
OP\_MAIOR\_IGUAL & \texttt{>=} & Operador relacional maior ou igual que. \\
OP\_AND & \texttt{\&\&} & Operador lógico E (AND). \\
OP\_OR & \texttt{||} & Operador lógico OU (OR). \\
OP\_NOT & \texttt{!} & Operador lógico de negação (NOT). \\
\midrule
\multicolumn{3}{l}{\textbf{Pontuadores e Delimitadores}} \\
\midrule
PONTO\_VIRGULA & \texttt{;} & Finalizador de instrução. \\
DOIS\_PONTOS & \texttt{:} & Usado em casos de `switch`. \\
VIRGULA & \texttt{,} & Separador de elementos (ex: em listas). \\
ABRE\_PARENTESES & \texttt{(} & Abre lista de parâmetros ou expressão. \\
FECHA\_PARENTESES & \texttt{)} & Fecha lista de parâmetros ou expressão. \\
ABRE\_CHAVES & \texttt{\{} & Abre um bloco de código. \\
FECHA\_CHAVES & \texttt{\}} & Fecha um bloco de código. \\
ABRE\_COLCHETES & \texttt{[} & Abre a declaração/acesso de um array. \\
FECHA\_COLCHETES & \texttt{]} & Fecha a declaração/acesso de um array. \\
\end{longtable}

\section{Análise de Erros}

A função \texttt{print\_error} exibe as mensagens de erro léxico e inclui a linha, coluna e token que causaram o erro. Para isto, utilizamos rastreamento de posição durante toda a leitura do código-fonte.

\subsection{Rastreamento de Posição}

Para tratar os erros léxicos de forma eficiente, o analisador léxico do Simple C implementa um mecanismo de rastreamento de posição. Para isto, são declaradas 2 variáveis globais, \texttt{current\_line} e \texttt{current\_col}, que armazenam a linha e a coluna atuais do analisador léxico, respectivamente. Essas variáveis são atualizadas conforme o analisador lê o código-fonte, permitindo que mensagens de erro incluam informações precisas sobre a localização do erro.

\section{Gerenciamento de Estados}

Para melhor gerenciar o estado do analisador léxico, utilizamos a funcionalidade de estados do Flex. Em especial, foram criados 2 estados diferentes: \texttt{IN\_COMMENT} e \texttt{IN\_STRING}. 

\subsection{Comentários}
Os comentários podem aparecer de 2 formas no código: (i) comentários de linha única, que começam com \texttt{//} e se estendem até o final da linha; (ii) comentários de bloco, que começam com \texttt{/*} e terminam com \texttt{*/}.

\subsection{Strings}

As strings são sequências de caracteres delimitadas por aspas duplas (\texttt{\"}). O analisador léxico reconhece strings e trata caracteres de escape, como \texttt{\textbackslash n} para nova linha e \texttt{\textbackslash t} para tabulação.

Para isso, foi-se utilizado a função \texttt{yymore()} do Flex, que permite concatenar a string lida em \texttt{yytext} com a próxima parte da string lida, até que o caractere de fechamento (\texttt{\"}) seja encontrado.

\section{Descrição da linguagem regular}

\section{Aceitador}

\section{Conclusão}

\end{document}
