\documentclass[12pt,a4paper]{article}

% Pacotes para o português.
\usepackage[brazilian,provide=*]{babel}
\usepackage[utf8]{inputenc}
\usepackage[T1]{fontenc}
\usepackage{graphicx}
\usepackage{url}
\usepackage{textcomp}
\usepackage{xcolor}
\usepackage{longtable}
\usepackage{indentfirst}
\usepackage{url}
\usepackage{array}
\usepackage[top=2.5cm, bottom=2.5cm, left=2.5cm, right=2.5cm]{geometry}
\usepackage{multirow}
\usepackage{amssymb}
\usepackage{amsmath}
\usepackage{caption}
\usepackage{setspace}
\usepackage{breakcites}
\usepackage{float}
\usepackage{times}
\usepackage{lipsum}
\usepackage{booktabs}
\usepackage{tikz}
\usepackage{amsmath}
\usetikzlibrary{shapes.multipart, arrows.meta, positioning, fit}

\usepackage{sectsty}
\usepackage[compact]{titlesec}

\sectionfont{\normalfont\normalsize\bfseries}
\subsectionfont{\normalfont\normalsize\bfseries}
\subsubsectionfont{\normalfont\normalsize\bfseries}

% Comando para marcar o texto para revisão.
\newcommand{\rev}[1]{\textcolor{red}{#1}}

% Permite escrever aspas normais "text" em vez de ``text''
\usepackage[autostyle]{csquotes}
\MakeOuterQuote{"}

\begin{document}

\begin{titlepage}
	\begin{center}
	
	\vspace{115pt}
 \textbf{\Huge{Descrição da Linguagem Regular}}\\
  
	\vspace{115pt}
 Felipe Gomes da Silva \\
 Luis Henrique Salomão Lobato \\
	\end{center}
	
	\vspace{1cm}
	\begin{center}
		\vspace{\fill}
 \large{Setembro, 2025} 
	\end{center}
\end{titlepage}


\section{Tokens Reconhecidos}
\label{sec:token}

O analisador léxico da linguagem Simple C toma a decisão de aceitar ou não determinada sentença com base nos tokens destacados na Tabela \ref{tab:tokens}

\begin{longtable}{lll}
\caption{Tabela de Tokens do Simple C}
\label{tab:tokens} \\

\toprule
\textbf{Nome do Token} & \textbf{Lexema(s)} & \textbf{Descrição} \\
\midrule
\endfirsthead

\toprule
\multicolumn{3}{l}{\small\textit{Tabela \ref{tab:tokens}: Tabela de Tokens do Simple C (Continuação)}} \\
\textbf{Nome do Token} & \textbf{Lexema(s)} & \textbf{Descrição} \\
\midrule
\endhead

\bottomrule
\endlastfoot

\multicolumn{3}{l}{\textbf{Palavras-chave: Tipos de Dados}} \\
\midrule
KEYWORD\_INT & \texttt{int} & Palavra-chave para tipo inteiro. \\
KEYWORD\_FLOAT & \texttt{float} & Palavra-chave para tipo ponto flutuante. \\
KEYWORD\_CHAR & \texttt{char} & Palavra-chave para tipo caractere. \\
KEYWORD\_STRING & \texttt{string} & Palavra-chave para tipo string (cadeia de caracteres). \\
KEYWORD\_VOID & \texttt{void} & Palavra-chave para tipo vazio/nulo. \\
KEYWORD\_BOOL & \texttt{bool} & Palavra-chave para tipo booleano. \\
\midrule
\multicolumn{3}{l}{\textbf{Palavras-chave: Controle de Fluxo e Comandos}} \\
\midrule
KEYWORD\_IF & \texttt{if} & Inicia uma estrutura condicional. \\
KEYWORD\_ELSE & \texttt{else} & Bloco alternativo de uma estrutura condicional. \\
KEYWORD\_FOR & \texttt{for} & Inicia um laço de repetição `for`. \\
KEYWORD\_WHILE & \texttt{while} & Inicia um laço de repetição `while`. \\
KEYWORD\_DO & \texttt{do} & Inicia um laço de repetição `do-while`. \\
KEYWORD\_SWITCH & \texttt{switch} & Inicia uma estrutura de seleção múltipla. \\
KEYWORD\_CASE & \texttt{case} & Define um rótulo dentro de um `switch`. \\
KEYWORD\_DEFAULT & \texttt{default} & Define o rótulo padrão de um `switch`. \\
KEYWORD\_BREAK & \texttt{break} & Interrompe a execução de um laço ou `switch`. \\
KEYWORD\_CONTINUE & \texttt{continue} & Pula para a próxima iteração de um laço. \\
KEYWORD\_RETURN & \texttt{return} & Retorna um valor de uma função. \\
\midrule
\multicolumn{3}{l}{\textbf{Identificadores e Literais}} \\
\midrule
IDENTIFICADOR & \texttt{var}, \texttt{\_x}, ... & Nome de variável, função, etc. \\
INT & \texttt{123}, \texttt{42} & Valor literal inteiro. \\
FLOAT & \texttt{3.14}, \texttt{0.5} & Valor literal de ponto flutuante. \\
CHAR & \texttt{'a'}, \texttt{'\textbackslash n'} & Valor literal de caractere. \\
STRING & \texttt{"hello"} & Valor literal de string. \\
BOOLEAN\_LITERAL & \texttt{true}, \texttt{false} & Valor literal booleano. \\
\midrule
\multicolumn{3}{l}{\textbf{Operadores}} \\
\midrule
OP\_SOMA & \texttt{+} & Operador de adição. \\
OP\_SUB & \texttt{-} & Operador de subtração. \\
OP\_MULT & \texttt{*} & Operador de multiplicação. \\
OP\_DIV & \texttt{/} & Operador de divisão. \\
OP\_MOD & \texttt{\%} & Operador de módulo (resto da divisão). \\
OP\_ATRIBUICAO & \texttt{=} & Operador de atribuição simples. \\
OP\_INC\_ATRIBUICAO & \texttt{+=} & Operador de atribuição com adição. \\
OP\_DEC\_ATRIBUICAO & \texttt{-=} & Operador de atribuição com subtração. \\
OP\_MULT\_ATRIBUICAO & \texttt{*=} & Operador de atribuição com multiplicação. \\
OP\_DIV\_ATRIBUICAO & \texttt{/=} & Operador de atribuição com divisão. \\
OP\_INC & \texttt{++} & Operador de incremento. \\
OP\_DEC & \texttt{--} & Operador de decremento. \\
OP\_IGUAL & \texttt{==} & Operador relacional de igualdade. \\
OP\_DIFERENTE & \texttt{!=} & Operador relacional de desigualdade. \\
OP\_MENOR & \texttt{<} & Operador relacional menor que. \\
OP\_MAIOR & \texttt{>} & Operador relacional maior que. \\
OP\_MENOR\_IGUAL & \texttt{<=} & Operador relacional menor ou igual que. \\
OP\_MAIOR\_IGUAL & \texttt{>=} & Operador relacional maior ou igual que. \\
OP\_AND & \texttt{\&\&} & Operador lógico E (AND). \\
OP\_OR & \texttt{||} & Operador lógico OU (OR). \\
OP\_NOT & \texttt{!} & Operador lógico de negação (NOT). \\
\midrule
\multicolumn{3}{l}{\textbf{Pontuadores e Delimitadores}} \\
\midrule
PONTO\_VIRGULA & \texttt{;} & Finalizador de instrução. \\
DOIS\_PONTOS & \texttt{:} & Usado em casos de `switch`. \\
VIRGULA & \texttt{,} & Separador de elementos (ex: em listas). \\
ABRE\_PARENTESES & \texttt{(} & Abre lista de parâmetros ou expressão. \\
FECHA\_PARENTESES & \texttt{)} & Fecha lista de parâmetros ou expressão. \\
ABRE\_CHAVES & \texttt{\{} & Abre um bloco de código. \\
FECHA\_CHAVES & \texttt{\}} & Fecha um bloco de código. \\
ABRE\_COLCHETES & \texttt{[} & Abre a declaração/acesso de um array. \\
FECHA\_COLCHETES & \texttt{]} & Fecha a declaração/acesso de um array. \\
\end{longtable}

\newpage

\section{Descrição da linguagem regular para o analisador léxico}
\label{sec:desc}

A linguagem regular estruturada para a aceitação de cadeias pelo analisador léxico da linguagem Simple-C pode ser descrita através das expressões regulares e definições presentes na Tabela \ref{tab:expressoes}. Deste modo, é garantido que variáveis não sejam declaradas com números como primeiro caractere da cadeia, além de outros erros léxicos devidamente discutidos.

\begin{table}[H]
\centering
\caption{Expressões regulares do analisador léxico.}
\label{tab:expressoes}
\begin{tabular}{l p{5cm} l}
\toprule
\textbf{Expressão Regular} & \textbf{Descrição} & \textbf{Tipo} \\
\midrule
\texttt{\detokenize{[a-zA-Z_][a-zA-Z0-9_]*}} & Reconhece identificadores válidos que começam com uma letra ou um sublinhado. & Identificador \\
\addlinespace
\texttt{\detokenize{0|[+-]?[1-9][0-9]*}} & Reconhece números inteiros. & Literal \\
\addlinespace
\texttt{\detokenize{[+-]?([0-9]+\.[0-9]*|\.[0-9]+)}} & Reconhece números de ponto flutuante. & Literal \\
\midrule
\texttt{\detokenize{//.*}} & Comentário de linha única, que começa com \texttt{//} e vai até o final da linha. & Comentário \\
\addlinespace
\texttt{\detokenize{/*}} & Inicia um comentário de bloco. & Comentário \\
\addlinespace
\texttt{\detokenize{<IN_COMMENT>"*/"}} & Fecha um comentário de bloco. & Comentário \\
\midrule
\texttt{\detokenize{"(\.|[^"\n])+"}} & Conteúdo de uma string, incluindo caracteres escapados. & String \\
\addlinespace
\texttt{\detokenize{'(\.|[^'\\])'}} & Reconhece literais de caracteres. & Caractere \\
\midrule
\texttt{\detokenize{==, !=, <=, >=, etc.}} & Operadores de comparação, lógicos e aritméticos. & Operador \\
\addlinespace
\texttt{\detokenize{;, <, >, {, }}} & Símbolos de pontuação e agrupamento. & Símbolo \\
\midrule
\texttt{\detokenize{{INT}{ID}}} & Captura identificadores que começam com um número, que é um erro léxico. & Erro \\
\addlinespace
\texttt{\detokenize{.}} & Captura qualquer caractere que não corresponda a nenhuma regra. & Erro \\
\texttt{\detokenize{[+-]?0[0-9]+}} & Captura números inteiros com zeros à esquerda, que é um erro léxico. & Erro \\
\addlinespace
\bottomrule
\end{tabular}
\end{table}

\end{document}
